\documentclass{beamer}

\usepackage{puremath}
\usepackage{minted}
\newminted{matlab}{mathescape, linenos, numbersep=5pt, frame=lines, framesep=2mm, fontsize=\tiny}
\usepackage{hyperref}
\usepackage{multimedia}

\setbeamercolor{note}{fg=black,bg=lightgray}

\title{PID Control for Quadcopter Stabilization}
\author{Andrew Gibiansky \\ \texttt{andrew.gibiansky@gmail.com}}
\date{\today}

\usetheme{Warsaw}

\begin{document}
\begin{frame}
    \titlepage
\end{frame}

\section{PD/PID Control}
\subsection{PD Control}
\begin{frame}
    \frametitle{System}
    Suppose we have some system with the transfer function
    \[G(s) = \frac{1}{a_0s^2+a_1s}.\]
    We would like this system to follow some desired trajectory $\theta$. However, 
    we have a disturbance (due to modeling error or external factors) summarized by a
    term $D(s)$. In addition, this system is clearly not asymptotically stable.
%    Suppose we have the following system:
%    \begin{align*}
%        \dot x &= \bmatr{
%            1 & 2 \\
%            3 & -4
%        } x + \bmatr{
%            0 \\ 1
%        } u \\
%        y &= \bmatr{1 & 0} x
%    \end{align*}
%    The transfer function is given by
%    \[H(s) = \frac{2}{(s-2)(s+5)}.\]
%    This system is unstable.
\end{frame}

\begin{frame}
    \frametitle{PD Control}
    We can drive the difference between the true trajectory and the desired trajectory to zero. The
    open loop system would be:
    \[\theta_t(s) = \theta(s) G(s).\]
    In order to drive the error to zero, we introduce a PD compensator:
    \[F(s) = K_p + K_d s.\]
    The new input is given by
    \[U(s) = F(s) (\theta_t(s) - \theta(s)).\]
    Then, the output is
    \begin{align*}
        \theta_t(s) &= (F(s) (\theta_t(s) - \theta(s)) + D(s)) G(s) \implies \\
        \theta_t(s) &= \frac{K_p\theta(s) - D(s) + K_d\theta(s)s}{K_p + K_ds - a_1s - a_0s^2}
    \end{align*}
\end{frame}

\begin{frame}
    \frametitle{Final Value Theorem}
    \textbf{Theorem:} Suppose $F(s)$ is the unilateral Laplace transform of $f(t)$, and $f(t)$
    converges to some value as $t \to \infty$. Then,
    \[\lim_{t\to\infty} f(t) = \lim_{s\to 0} sF(s).\]
\end{frame}

\begin{frame}
    \frametitle{PD Control Results}
    Suppose we have a step trajectory
    \[\theta(s) = \cL\left\{ T \right\} = \frac{T}{s}\]
    and a similar constant disturbance term
    \[D(s) = \cL\left\{ D \right\} = \frac{D}{s}\]
    Then, the tracking error is
    \begin{align*}
        E(s) &= \theta(s) - \theta_t(s) \\
        &= \frac{a_0Ts^2 + Ta_1s - D}{s(K_p + K_ds - a_1s - a_0s^2)}
    \end{align*}
\end{frame}

\begin{frame}
    \frametitle{Steady State Error}
    By the final value theorem:
    \[\lim_{t\to\infty} e(t) = \lim_{s\to 0} sE(s) = -\frac{D}{K_p}.\]
    \centerline{\textbf{Steady state error!}}
    In physical systems, we cannot make the gain $K_p$ arbitrarily large, so
    this becomes a real problem. A PD controller cannot accurately track a trajectory with a step in
    it.\\
    (The error comes from the fact that eventually the system has zero derivative error, so $K_d$ is
    ineffective, and the constant disturbance balances out the proportional gain $K_p$.)
\end{frame}

\subsection{PID Control}
\begin{frame}
    \frametitle{PID Control}
    Add an integral term:
    \[F(s) = K_p + K_d s + K_i/s.\]
    This integral term ensures that if we remain in a steady state error for long enough, the
    build-up will overcome any constant disturbance. The error becomes:
    \[E(s) = \frac{Ta_0s^2 + Ta_1s - D}{K_i + K_ps + K_ds^2 - a_0s^3 - a_1s^2}\]
    Final value theorem:
    \[\lim_{t\to\infty} e(t) = \lim_{s\to 0} sE(s) = 0.\]
\end{frame}

\subsection{General Case}
\begin{frame}
    \frametitle{General Case: PD Control}
    Suppose we have a system with transfer function $H(s)/s$, such that $H(0) \ne 0$. Then, using a
    PD controller where $F(s) = K_p + K_d s$ we have
    \begin{align*}
        E(s) &= (F(s) E(s) + D(s)) H(s)/s - \theta(s) \\
        E(s) &= \frac{F(s)H(s)/s + D(s)H(s)/s - \theta(s)}{1-F(s)H(s)/s} \\
        \hline\\
        \lim_{t\to\infty} e(t) &= \lim_{s\to 0} s E(s)
        = \lim_{s\to 0} s\frac{F(s)H(s) + D(s)H(s) - s\theta(s)}{s-F(s)H(s)} \\
        &= \lim_{s\to 0} \frac{sF(s)H(s) + s(D/s)H(s) - s^2\theta(s)}{s-F(s)H(s)}
        = -\frac{D}{K_p} \\
    \end{align*}
    We have the steady state error shown before.
\end{frame}

\begin{frame}
    \frametitle{General Case: PID Control}
     With a PID control with $F(s) = K_p + K_d s + K_i/s$,
    we obtain the same form of the result:
    \begin{align*}
        \lim_{t\to\infty} e(t) &= \lim_{s\to 0} \frac{sF(s)H(s) + s(D/s)H(s) - s^2\theta(s)}{s-F(s)H(s)} \\
        &= \lim_{s\to 0} \frac{D}{-K_p + K_d s + K_i/s} = 0 \\
    \end{align*}
    Thus, the result we found before generalize for any transfer function of the form $H(s)/s$ where
    $H(0) \ne 0$.
\end{frame}

\section{Quadcopters}

\subsection{Mechanical View}
\begin{frame}
    \frametitle{Quadcopters}
    \begin{center}
        \gfx{imgs/presentation/quadcopter.jpg}{0.25} 
        \gfx{imgs/presentation/quadcopter2.jpg}{0.25} \\
    \end{center}
\end{frame}

\begin{frame}
    \frametitle{Quadcopter Thrusts}
    \begin{center}
        \gfx{imgs/presentation/quad_physics1.jpg}{0.55} 
        \gfx{imgs/presentation/quad_physics2.jpg}{0.55} \\
        Degrees of freedom provided by extra motors.
    \end{center}
\end{frame}

\subsection{Kinematics}
\begin{frame}
    \frametitle{Coordinate Systems}
    \begin{figure}[h]
        \cgfx{imgs/Quadcopter_Coordinates.png}{0.3}
        \caption{Quadcopter Body Frame and Inertial Frame}
    \end{figure}
\end{frame}

\begin{frame}
    \frametitle{Forces and Torques}
    \begin{itemize}
        \item Thrust force:
            \[T_B = \sum_{i=1}^4 T_i = k\bmatr{ 0 \\ 0 \\ \sum {\omega_i}^2 }.\]
        \item Drag force:
            \[F_D = \bmatr{
                -k_d\dot x \\
                -k_d\dot y \\
                -k_d\dot z \\
            }\]
        \item Motor torques:
            \[\tau_B = \bmatr{
                Lk({\omega_1}^2 - {\omega_3}^2) \\
                Lk({\omega_2}^2 - {\omega_4}^2) \\
                b\left( {\omega_1}^2 -  {\omega_2}^2 +  {\omega_3}^2 -  {\omega_4}^2\right)
            }\]
    \end{itemize}
\end{frame}

\subsection{Equations of Motion}
\begin{frame}
    \frametitle{Equations of Motion: Linear}
    \begin{itemize}
        \item Inertial Frame:
            \[m\ddot{x} = \bmatr{0 \\ 0 \\ -mg} + RT_B + F_D\]
        \item Body Frame:
            \[m\ddot{x} = R^{-1}\bmatr{0 \\ 0 \\ -mg} + T_B + R^{-1}F_D - \omega\times(m\dot x)\]
    \end{itemize}
\end{frame}

\begin{frame}
    \frametitle{Equations of Motion: Rotational}
    \begin{itemize}
        \item Euler's Equations (Rigid Body Dynamics):
            \[I\dot\omega + \omega\times (I\omega) = \tau\]
        \item Body Frame Dynamics:
            \[I = \bmatr{I_{xx} & 0 & 0 \\ 0 & I_{yy} & 0 \\ 0 & 0 & I_{zz}}.\]
            \[\dot\omega = \bmatr{
                \tau_\phi {I_{xx}}^{-1} \\
                \tau_\theta {I_{yy}}^{-1} \\
                \tau_\psi {I_{zz}}^{-1}
            } - \bmatr{
                \frac{I_{yy} - I_{zz}}{I_{xx}} \omega_y\omega_z \\ 
                \frac{I_{zz} - I_{xx}}{I_{yy}} \omega_x\omega_z  \\
                \frac{I_{xx} - I_{yy}}{I_{zz}} \omega_x\omega_y
            }\]
    \end{itemize}
\end{frame}

\begin{frame}[fragile]
    \frametitle{Simulation}
\begin{matlabcode}
% Simulation times, in seconds.
start_time = 0; end_time = 10; dt = 0.005;
times = start_time:dt:end_time;
N = numel(times);

% Initial simulation state.
x = [0; 0; 10]; xdot = zeros(3, 1); theta = zeros(3, 1);

% Simulate some disturbance in the angular velocity.
% The magnitude of the deviation is in radians / second.
deviation = 100;
thetadot = deg2rad(2 * deviation * rand(3, 1) - deviation);

% Step through the simulation, updating the state.
for t = times
    i = input(t);                                          % Get controller input
    omega = thetadot2omega(thetadot, theta);               % Convert $(\dot\phi, \dot\theta, \dot\psi) \to \dot\omega$
    a = acceleration(i, theta, xdot, m, g, k, kd);         % Compute linear acceleration
    omegadot = angular_acceleration(i, omega, I, L, b, k); % Compute angular acceleration
    omega = omega + dt * omegadot;                         % $\omega = \omega + dt\times\dot\omega$
    thetadot = omega2thetadot(omega, theta);               % Convert $\omega \to (\dot\phi, \dot\theta, \dot\psi)$
    theta = theta + dt * thetadot;                         % $\vec\theta = \vec\theta + dt\times(\dot\phi, \dot\theta, \dot\psi)$
    xdot = xdot + dt * a;                                  % $\dot{x} = \dot{x} + dt\times a$
    x = x + dt * xdot;                                     % $x = x + dt\times\dot{x}$
end
\end{matlabcode}
\end{frame}

\begin{frame}
    \frametitle{Visualization}
    \cgfx{imgs/presentation/simulation.png}{0.22}
    \begin{center}
        \href{http://www.youtube.com/watch?v=cVrM1iAd8ic}{YouTube Movie}
    \end{center}
\end{frame}

\section{Control}
\subsection{PD Control}
\begin{frame}
    \frametitle{PD Control}
    \begin{itemize}
        \item Control signal should be turned into a torque:
            \[\tau = I u(t) = I\dot\omega\]
        \item We can set torques:
            \[\bmatr{\tau_\phi \\ \tau_\theta \\ \tau_\psi} = \bmatr{
                -I_{xx} \left( K_d\dot\phi + K_p \int_0^T \dot\phi \dd t \right) \\
                -I_{yy} \left( K_d\dot\theta + K_p \int_0^T \dot\theta \dd t \right) \\
                -I_{zz} \left( K_d\dot\psi + K_p \int_0^T \dot\psi \dd t \right) \\
            }\]
        \item Our electronic gyro outputs angular velocities, so we integrate them to get the angle.
            This integral is multiplied by the  \emph{proportional} gain.
    \end{itemize}
\end{frame}

\begin{frame}
    \frametitle{Motor Angular Velocities}
    \begin{itemize}
        \item We do not actually set torques, we set voltages over the motors. These correspond
            directly to the motor angular velocities $\omega_i$.

        \item We can solve for the inputs $\gamma_i = {\omega_i}^2$:
            \[\tau_B = \bmatr{
                Lk({\gamma_1} - {\gamma_3}) \\
                Lk({\gamma_2} - {\gamma_4}) \\
                b\left( {\gamma_1} -  {\gamma_2} +  {\gamma_3} -  {\gamma_4}\right)
            } = \bmatr{
                -I_{xx} \left( K_d\dot\phi + K_p \int_0^T \dot\phi \dd t \right) \\
                -I_{yy} \left( K_d\dot\theta + K_p \int_0^T \dot\theta \dd t \right) \\
                -I_{zz} \left( K_d\dot\psi + K_p \int_0^T \dot\psi \dd t \right) \\
            }\]
        \item For the last constraint, choose one to keep the quadcopter aloft:
            \[T = \frac{mg}{\cos\theta\cos\phi}\]
    \end{itemize}
\end{frame}

\begin{frame}
    \frametitle{PD Control}
    \begin{figure}[h]
        \cgfx{imgs/pd_controller.png}{0.43}
        \caption{
            Left: Angular velocities. Right: angular displacements ($\phi$, $\theta$, $\psi$).
            Note that there is some small steady-state error (approximately 0.3$^\circ$).
        }
    \end{figure}
\end{frame}

\subsection{PID Control}
\begin{frame}
    \begin{itemize}
        \item We can do better by adding an integral term to make this a PID controller.
            \begin{align*}
                e_\phi &= K_d\dot\phi + K_p \int_0^T\dot\phi\dd t + K_i \int_0^T\int_0^T\dot\phi\dd t\dd t \\
                e_\theta &= K_d\dot\theta + K_p \int_0^T\dot\theta\dd t + K_i \int_0^T\int_0^T\dot\theta\dd t\dd t \\
                e_\psi &= K_d\dot\psi + K_p \int_0^T\dot\psi\dd t + K_i \int_0^T\int_0^T\dot\psi\dd t\dd t \\
            \end{align*}
        \item To avoid integral wind-up, only start integrating the double-integral once the angle
            deviation is relatively small.
    \end{itemize}
\end{frame}

\begin{frame}
    \frametitle{Integral Windup}
    \begin{figure}[H]
        \cgfx{imgs/windup.png}{0.4}
        \caption{
            In some cases, integral wind-up can cause lengthy oscillations instead of settling. In other cases,
            wind-up may actually cause the system to become unstable, instead of taking longer to reach
            a steady state.
        }
    \end{figure}
\end{frame}

\begin{frame}
    \frametitle{PID Control}
    \begin{figure}[H]
        \cgfx{imgs/pid_controller.png}{0.45}
        \caption{
            With a properly implemented PID, we achieve an error of approximately 0.06$^\circ$ after 10 seconds.
        }
    \end{figure}
\end{frame}

\subsection{Automatic Gain Tuning}
\begin{frame}
    \frametitle{Automatic Gain Tuning}
    \begin{itemize}
        \item Tuning the PID gains ($K_i$, $K_p$, and $K_d$) can be difficult.
        \item Quality of results depends on gain values and the ratios of the different gain values.
        \item ``Best'' gains might be different for different modes of operation.
        \item Requires expert intuition and a lot of time to tune them.
        \item We would like to do this automatically.
    \end{itemize}
\end{frame}

\begin{frame}
    \frametitle{Extremum Seeking}
    \begin{itemize}
        \item Define a cost function, defining the quality of a set of parameters:
            \[J(\vec\theta) = \inv{t_f - t_o} \int_{t_0}^{t_f} e(t, \vec\theta)^2 \dd t\]
        \item Use gradient descent to minimize this cost function in parameter-space:
            \[\vec\theta(k + 1) = \vec\theta(k) - \alpha \nabla J(\vec \theta)\]
        \item Approximate gradient numerically:
            \[\nabla J(\vec\theta) = \left(\p{ }{K_p} J(\vec\theta), \p{ }{K_i} J(\vec\theta), \p{ }{K_d} J(\vec\theta)\right).\]
            \[\p{ }{K} J(\vec\theta) \approx \frac{J(\vec\theta + \delta\cdot \hat{u}_K) - J(\vec\theta- \delta\cdot \hat{u}_K)}{2\delta}\] 
    \end{itemize}
\end{frame}

\begin{frame}
    \frametitle{Gradient Descent Tricks}
    \begin{itemize}
        \item Adjust step size $\alpha$ as we go along, to become more precise as time goes on.
        \item Computing $e(t, \vec\theta)$ means running a simulation with some random initial
            disturbance. Use many simulations, take the average. As we go along, average more
            simulations.
        \item Repeat many times to produce many local minima, then choose the best one and call it
            the ``global minimum''.
        \item Automatically choose a time to stop iterating (when we're no longer improving our
            average cost).
    \end{itemize}
\end{frame}

\begin{frame}
    \frametitle{Manual Gains vs. Automatically Tuned Gains}
    \begin{figure}[H]
        \cgfx{imgs/pid_gains_comparison.png}{0.51}
    \end{figure}
\end{frame}


\begin{frame}
    \frametitle{Questions?}
    \begin{center}
        \scriptsize
        Thank you to Professors Donatello Materassi and Rob Wood for their help.
    \end{center}
\end{frame}


\end{document}
